% !TEX encoding = UTF-8 Unicode
\documentclass{article}

\usepackage{polski}
\usepackage[utf8]{inputenc}
\usepackage{subfig}
\usepackage{graphicx}

\usepackage[a4paper, left=2.5cm, right=2.5cm, top=3.5cm, bottom=3.5cm, headsep=1.2cm]{geometry}

\linespread{1.3}
\begin{document}
	
	\begin{titlepage}
		\centering
		{\scshape\LARGE Politechnika Wrocławska \par}
		{\scshape\Large Katedra Systemów i Sieci Komputerowych \par}
		
		\vspace{1cm}
		{\scshape\Large Technologie Sieciowe 2\par}
		\vspace{5cm}
		{\huge\bfseries Projekt przedmiotowy\par}
		\vspace{5cm}
		{\Large\itshape Magdalena Biernat, 225934\par}
		{\Large\itshape Michał Duński, 226081\par}
		\vfill
		Opiekun\par
		dr inż. Michał Kucharzak 
		
		\vfill
		{\large \today\par}
	\end{titlepage}
	\newpage
	\section{Wstęp}
	 Zadaniem tego projektu jest zaprojektowanie sieci komputerowej dla firmy RoboNet - przedsiębiorstwa zajmującego się produkcją oprogramowania dla specjalistycznych urządzeń ‒ robotów. Firma zatrudnia ok. 180 osób podzielonych na 3 grupy robocze, które zajmują 2 budynki. Budynek A posiada 3 kondygnacje, Budynek B posiada 2 kondygnacje. Laboratorium znajduje się na parterze w budynku A. Sieć laboratoryjna nie ma dostępu do internetu. Do sieci laboratoryjnej mają dostęp wyłącznie Programiści i Testerzy. Serwery plików, www i pocztowy znajdują się w Budynku A i mieszczą się na dwóch kondygnacjach. Jeden serwer jest umieszczony w Budynku B.\newline
	 \noindent
	 \newline
Planujemy zastosować odpowiednie programy antywirusowe dla bezpieczeństwa oprogramowania oraz aby ograniczyć dostęp do sieci.
\newline
\noindent
\newline
	Projektowana sieć powinna cechować się jakością, niezawodnością oraz skalowalnością w przypadku potrzeby zwiększenia ilości pracowników w firmie. Ważnym czynnikiem jest również estetyczna jakość wykonania instalacji.
\newpage
\section{Inwentaryzacja zasobów: sprzętu, aplikacji, zasobów ludzkich}
Siedziba firmy mieści się w dwóch budynkach o oznaczeniach A i B. Budynek A jest trzypiętrowy, a budynek B ma tylko parter. 
\subsection{Wykaz pomieszczeń w budynkach}
\begin{enumerate}
	\item Budynek A
	\begin{itemize}
		\item Parter: administratorzy, serwerownia 1
		\item Piętro 1: programiści i testerzy, serwerownia pocztowa, serwerownia ww
		\item Piętro 2: zarząd i kadry, programiści i testerzy
	\end{itemize}
	\item Budynek B
		\begin{itemize}
		\item Parter: zarząd i kadry, programiści i testerzy, serwerownia 2
	\end{itemize}
\end{enumerate}
\subsection{Plany}
\subsection{Sprzęt}
Firma na wyposażeniu posiada:
\begin{itemize}
	\item 16 robotów
	\item 7 drukarek
	\item 24 kamery IP
\end{itemize}
\begin{tabular}{|r|r|r|r|r|}\hline
	& Budynek A & Budynek A & Budynek A & 	Budynek B 	\\
	\hline
 & parter & piętro I & pietro II & parter \\
 \hline
drukarki & 1 & 2 & 2 & 2\\
\hline
roboty (urządzenia) & 16 & - & - & - \\
\hline
kamery IP & 8 & 4 & 4 & 8 \\
\hline
\end{tabular}
\section{Analiza potrzeb użytkowników – wymagania zamawiającego}
\section{Założenia projektowe}
Projekt zakłada stworzenie sieci dla przedsiębiorstwa zajmującego się produkcją oprogramowania dla specjalistycznych urządzeń ‒ robotów, których zastosowanie jest ściśle tajne. Przedsiębiorstwo posiada dwa budynki. W jednym pracuje 100 użytkowników (komputerów), 5 drukarek, 16 kamer IP, 16 robotów i 3 serwery. W drugim pracuje 80 użytkowników (komputerów, 2 drukarki, 8 kamer IP i 1 serwer. W każdym budynku projekt zakłada sieć WiFi dla 150 gości. Budynek A ma trzy kondygnacje, budynek B posiada tylko parter. Przed stworzeniem sieci komputerowej zostanie wykonane (we wcześniejszym terminie i dla odpowiednich pomieszczeń) dostosowanie instalacji elektrycznej.
W obu budynkach będą znajdować się przełączniki warstwy trzeciej.
Dla połączenia z Internetem zostaną zamontowany router chroniony firewallem.
Z sieci gości możliwy jest wyłącznie dostęp do Internetu.
Wszyscy pracownicy mają dostęp do wszystkich drukarek i pozostałych serwerów. Z Internetu możliwy jest dostęp wyłącznie do Serwera WWW i Serwera Pocztowego.
Okablowanie poziome w technologii 100BASE-TX, okablowanie pionowe
Okablowanie poziome w technologii 100Base-TXFast, okablowanie pionowe w technologii 1000Base-T Gigabit Ethernet oraz połączenie światłowodowe między budynkami. Dla zachowania odpowiedniej estetyki kable zostaną schowane w podłodze lub podwieszanym suficie.
Zastosowanie odpowiednich programów antywirusowych dla bezpieczeństwa oprogramowania oraz ograniczony dostęp do sieci.

\end{document}}